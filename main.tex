\documentclass[a4paper]{article}
\usepackage{amsmath, amssymb, amsthm, graphicx, epsfig, fancyhdr, enumitem, breqn, bm, geometry, lastpage}

\usepackage[english]{babel}
\usepackage[utf8x]{inputenc}
\usepackage[colorinlistoftodos]{todonotes}
\usepackage[affil-it]{authblk}
\theoremstyle{example}
\newtheorem{example}{Example}


\title{Math Test Tips from the CWLT}
\author{Matthew Moreno}
\usepackage{authblk}
\begin{document}
\maketitle

\section{During the Test}

\begin{itemize}
	\item try to do work on separate page (double check if you copy the problem over)
    \item only do one step at a time --- showing more work makes you less likely to make errors and gives you more opportunity for partial credit
    \begin{example}[One Step at a Time]
    \end{example}
    \item Do the problems you know how to do right away first; problems you are less sure of will be less intimidating if you don't have to worry about missing out on problems that you know how to do!
    \item mark problems and steps that you are unsure of with a big $\circ$; return to these at the end of the test if you have time

\end{itemize}
\section{If You Have Extra Time...}
\begin{itemize}
    \item return to the problems you are unsure of or skipped that are marked with a  $\circ$
    \item use your calculator to double check your work
   	\begin{itemize}
    	\item plug in solutions and verify they work
            \begin{example}[Plugging in a Solution]
   			\end{example}
        \item plug in random values into expressions before/after you manipulated them and verify that you got the same thing out
    	\begin{example}[Verify an Expression]
   			\end{example}
		\item double check the result from you calculator; more often than not, mistakes are made checking your work than doing the work
	\end{itemize}

\end{itemize}
\section{If You Still Have Extra Time...}
\begin{itemize}
    \item if you have extra time, ink in the main flow of logic in your answers! This is a great way to systematically double-check your work and your exam will look very nice!
    	\begin{example}[Inking in your Test]
   		\end{example}
    \item make sure that your answers are the type and format of answers requested by the professor
    \begin{example}[Answer Format]
   	\end{example}
\end{itemize}
\end{document}